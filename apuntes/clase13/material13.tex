\begin{framed}

Objetivos:
\begin{itemize}
    \item Repasar el rol de la ecuación de estado en mecánica de fluidos.
    \item Repasar conceptos de termodinámica relevantes en flujos compresibles.
    \item Estudiar la propagación de ondas de presión en flujos compresibles.
\end{itemize}

Contenidos:
\begin{itemize}
    \item Ecuaciones de conservación en mecánica de fluidos para flujos incompresibles y compresibles. 
    \item La ecuación de estado para gases ideales. 
    \item Constantes termodinámicas.
    \item Energía interna, entalpía y entropía.
    \item Procesos isentrópicos.
    \item Ondas de presión
    \item El cono de Mach. 
\end{itemize}

Bibliografía:
\begin{itemize}
    \item Fox, R. W., Pritchard, P. J. y McDonald, A. T. (2009) Introduction to Fluid Mechanics. John Wiley \& Sons. Sección 9.6-9.7.
    \item White, F. M. (2008) Mecánica de Fluidos. McGraw-Hill. Sexta edición. Secciones 7.5-7.6.
\end{itemize}
\end{framed}

\section*{Ecuaciones de conservación en mecánica de fluidos}

Hasta ahora, hemos limitado nuestros análisis a flujos incompresibles.
A pesar de que ningún fluido es realmente incompresible (pues implicaría una velocidad del sonido infinita), para líquidos y gases a baja velocidad, esto es una buena aproximación.
En esta parte del curso vamos a tratar flujos compresibles, donde la variación de densidad juega un rol importante.
Una ventaja de los flujos incompresibles, es que hay ecuaciones que se esconden un poco de nosotros.
De hecho, estamos acostumbrados a tratar con la ecuación de continuidad y Navier-Stokes solamente:
%
\begin{align}\label{eq:incompresible}
\nabla\cdot\mathbf{V} &= 0\nonumber\\
\frac{D\mathbf{V}}{D t} &=-\frac{\nabla p}{\rho} +\nabla^2\mathbf{V}.
\end{align}
%
Sin embargo, también está la ecuación de conservación de la energía
%
\begin{equation}\label{eq:energia_incomp}
\rho c_p\frac{DT}{Dt} = -\nabla\cdot (\kappa\nabla T)
\end{equation}
%
donde $T$ es la temperatura, $c_p$ el calor específico a presión constante, y $\kappa$ el coeficiente de conductividad del medio.
La razón por la que comúnmente no operamos con la Ec. \eqref{eq:energia_incomp} es que para el caso incompresible no hay cambio de densidades debido a la temperatura, lo que desacopla la conservación de energía de la de momentum, y, a menos que queramos calcular el campo de temperaturas, no necesitamos la Ec. \eqref{eq:energia_incomp}.
Además, implícitamente hemos usado la ecuación de estado
%
\begin{equation}\label{eq:estado_incomp}
\rho = \text{constante}
\end{equation}
%
En general, necesitamos las cuatro ecuaciones para tener un sistema cerrado y modelar mecánica de fluidos, sin embargo, hasta ahora nos hemos focalizado en el caso fácil, donde necesitamos solamente dos de esas ecuaciones.

Volvamos un poco más atrás en las derivaciones para ver por qué necesitamos las cuatro ecuaciones.
En general, la ecuaciones de conservación de masa, momentum y energía son:
%
\begin{align}\label{eq:conservacion}
\frac{\partial\rho}{\partial t} + \nabla\cdot(\rho\mathbf{V})&=0\nonumber\\
\rho\frac{D\mathbf{V}}{Dt} &= -\nabla p + \nabla\cdot\tau_{ij}\nonumber\\
\rho\frac{D}{Dt}\left(u+\frac{p}{\rho}\right) &= \frac{Dp}{Dt}+\nabla\cdot(\kappa\nabla T) + \tau_{ij}\frac{\partial u_i}{\partial x_j}
\end{align}
%
donde $\tau_{ij}$ es el tensor con los esfuerzos de corte y $u$ la energía interna específica.
Si se fijan en la Ec. \eqref{eq:conservacion} tenemos 9 variables a calcular: $\rho$, $\mathbf{V}=(u,v,w)$, $p$, $\tau_{ij}$, $u$, $\kappa$ y $T$, sin embargo, solamente hay 5 ecuaciones (una de momentum por cada dirección, más energía y continuidad).
Para cerrar el sistema, necesitamos 4 ecuaciones más.
Una de estas ecuaciones las podemos sacar de relaciones constitutivas, o sea, ecuaciones propias del material. 
Por ejemplo, al asumir un fluido Newtoniano encontramos una relación entre la velocidad y $\tau_{ij}$ (de hecho, la usamos para llegar a la ecuación de Navier-Stokes).
El resto de las ecuaciones serán ecuaciones de estado, y las sacaremos de termodinámica.

Vamos al caso que nos interesa a nosotros.
En este curso, veremos flujos donde la compresibilidad es dominante frente a efectos viscosos, y asumiremos $\tau_{ij}=0$.
De esta forma, la Ec. \eqref{eq:conservacion} queda:
%
\begin{align}\label{eq:conservacion_novisc}
\frac{\partial\rho}{\partial t} + \nabla\cdot(\rho\mathbf{V})&=0\nonumber\\
\rho\frac{D\mathbf{V}}{Dt} &= -\nabla p \nonumber\\
\rho\frac{D}{Dt}\left(u+\frac{p}{\rho}\right) &= \frac{Dp}{Dt}+\nabla\cdot(\kappa\nabla T) 
\end{align}
%
donde la conservación de cantidad de movimiento nos llevó a la ecuación de Euler.
Para cerrar el systema, necesitamos dos ecuaciones de estado.
Afortunadamente de termodinámica sabemos que con dos parámetros termodinámicos se define un estado, por lo tanto la energía interna es función de dos variables intependientes, por ejemplo $u=u(v_e,T)$, donde $v_e=1/\rho$ es el volúmen específico.
Por esto, podemos escribir
%
\begin{equation}
du = \left(\frac{\partial u}{\partial T}\right)_vdT + \left(\frac{\partial u}{\partial v_e}\right)_Tdv_e
\end{equation}
%
Aquí aparece una definición que debiesen reconocer:
%
\begin{equation}
c_v = \left(\frac{\partial u}{\partial T}\right)_v
\end{equation}
%
que se conoce como el calor específico a volumen constante.
Quizás se acuerden en palabras: el calor específico es la cantidad de energía necesaria para subir un grado la temperatura.
